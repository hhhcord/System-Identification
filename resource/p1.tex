\textbf{Computational Steps} \\

To better understand the computational procedure for the SRIM algorithm, the computational steps are summarized as follows: 

\begin{enumerate}
  \item Choose an integer $p$ so that 
  \[
    p\geq\frac{n}{m}+1,
  \]
  where $n$ is the desired order of the system and $m$ is the number of outputs. 
  \item Compute correlation matrices $\mathcal{R}_{yy}$ of dimension $pm\times pm$, $\mathcal{R}_{yu}$ of dimension $pm\times pr$, and $\mathcal{R}_{uu}$ of dimension $pr\times pr$ (eqs. (14)) with the matrices $Y_p(k)$ of dimension $pm\times N$ and $\mathcal{U}_p(k)$ of dimension $pr\times N$ (eqs. (13)). 
  The integer $r$ is the number of inputs. 
  The index $k$ is the data point used as the starting point for system identification. 
  The integer $N$ must be chosen so that 
  \[
    \ell-k-p+2\geq N>>\min(pm,pr),
  \]
  where $\ell$ is the length of the data. 
  \item Calculate the correlation matrix $\mathcal{R}_{hh}$ of dimension $pm\times pm$ (eqs. (14)), that is, 
  \[
    \mathcal{R}_{hh}=\mathcal{R}_{yy}-\mathcal{R}_{yu}\mathcal{R}^{-1}_{uu}\mathcal{R}^\top_{yu}.
  \]
  \item Factor $\mathcal{R}_{hh}$ with singular-value decomposition for the full decomposition method (eq. (26)) or a portion of $\mathcal{R}_{hh}$ for the partial decomposition method (eq. (31)). 
  \item Determine the order $n$ of the system by examining the singular values of $\mathcal{R}_{hh}$, and obtain $\mathcal{U}_n$ of dimension $pm\times n$ (eq. (26)) and $\mathcal{U}_o$ of dimension $pm\times n_o$, where $n_o=pm-n$ is the number of truncated small singular values. 
  The integer $n_o$ must satisfy the condition 
  \[
    pn_o\geq(m+n)
  \]
  for the full decomposition method. 
  For the partial decomposition method, $\mathcal{U}_n$ is replaced by $\mathcal{U}'_n$ (eq. (31)) and the integer $n_o$ is the sum of $m$ and the number of truncated singular values. 
  \item Let $\mathcal{U}_n=\mathcal{O}_p$ or $\mathcal{U}'_n=\mathcal{O}_p$. 
  Use equation (8) to determine the state matrix $A$. 
  The output matrix $C$ is the first $m$ rows of $\mathcal{U}_n$. 
  \item Compute 
  \[
    \mathcal{U}_{o\mathcal{R}}=\mathcal{U}_o^\top\mathcal{R}_{yu}\mathcal{R}_{uu}^{-1}
  \]
  (eq. (39)) for the indirect method, and construct $\mathcal{U}_{on}$ and $\mathcal{U}_{o\mathcal{T}}$ (eqs. (38) and (40)). 
  Determine the input matrix $B$ and the direct transmission matrix $D$ from equation (41) (i.e., the first $m$ rows of $\mathcal{U}_{on}^\dagger \mathcal{U}_{o\mathcal{T}}$ from $D$ and the last $n$ rows produce $B$). 
  For the direct method, construct $\mathcal{O}_{p\Gamma}$ and $\mathcal{O}_{pA}$ from equation (53) and solve for matrices $B$ and $D$ by computing $\mathcal{O}_{pA}^\dagger \mathcal{O}_{p\Gamma}$. 
  The first $m$ rows of $\mathcal{O}_{pA}^\dagger\mathcal{O}_{p\Gamma}$ form matrix $D$, and the last $n$ rows produce matrix $B$. 

  For the output-error minimization method, construct $y_N(0)$ and $\Phi$ from equations (64) and solve for matrices $B$ and $D$ by computing $\Phi^\dagger y_N(0)$. 
  The first $n$ elements of $\Phi^\dagger y_N(0)$ form the initial state vector $x(0)$, the second $mr$ elements give the $r$ column vectors of $D$, and the last $nr$ elements produce the $r$ column vectors of $B$. 
  \item Find the eigenvalues and eigenvectors of the realized state matrix and transform the realized model into model coordinates to compute system damping and frequencies. 
  This step is needed only if modal parameters identification is desired. 
  \item Calculate mode singular values (ref. 1) to quantify and distinguish the system and noise modes. 
  This step provides a way for model reduction with modal truncation. 
\end{enumerate}

The computational steps reduce to the steps for the ERA/DC method (ref. 1) when the output data are the pulse-response-time history. 
Assume that a pulse is given to excite the system at the time step zero. 
Let $k=1$ in step 2. 
The correlation matrix $\mathcal{R}_{yu}$ and $\mathcal{R}_{uu}$ become null, and $\mathcal{R}_{hh}=\mathcal{R}_{yy}$ is obtained. 
Theoretically, the formulation 
\[
  \mathcal{R}_{hh}=\mathcal{R}_{yy}-\mathcal{R}_{yu}\mathcal{R}_{uu}^{-1}\mathcal{R}_{yu}^\top
\]
should not be used for computation of $\mathcal{R}_{hh}$ if $\mathcal{R}_{uu}$ is not invertible. 
For special cases such as free decay and pulse responses, $\mathcal{R}_{hh}$ reduces to $\mathcal{R}_{yy}$ when the integer $k$ is chosen at the point where the input signal vanishes. 