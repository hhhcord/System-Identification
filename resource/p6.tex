The colon in place of a subscript denotes the entire corresponding row or column. 
The state matrix can then be computed by 
\[
    A = \mathcal{O}_p^\dagger(1:(p-1)m,:)\mathcal{O}_p(m+1:pm,:)\quad(8)
\]

To determine $\mathcal{O}_p$ and $\mathcal{T}_p$, first expand the vector equation (eq. (5)) to a matrix equation as follows: 
\[
    Y_p(k)=\mathcal{O}_pX(k)+\mathcal{T}_pU_p(k)\quad(12)
\]

where
\begin{align*}
    X(k)&=\begin{bmatrix}
        x(k)&x(k+1)&\cdots&x(k+N-1)
    \end{bmatrix}\\
    Y_p(k)&=\begin{bmatrix}
        y_p(k)&y_p(k+1)&\cdots &y_p(k+N-1)
    \end{bmatrix}\\
    &=\begin{bmatrix}
        y(k)&y(k+1)&\cdots &y(k+N-1)\\
        y(k+1)&y(k+2)&\cdots &y(k+N)\\
        \vdots&\vdots&&\vdots\\
        y(k+p-1)&y(k+p)&\cdots &y(k+p+N-2)
    \end{bmatrix}\\
    U_p(k)&=\begin{bmatrix}
        u_p(k)&u_p(k+1)&\cdots&u_p(k+N-1)
    \end{bmatrix}\\
    &=\begin{bmatrix}
        u(k)&u(k+1)&\cdots &u(k+N-1)\\
        u(k+1)&u(k+2)&\cdots &u(k+N)\\
        \vdots&\vdots&&\vdots\\
        u(k+p-1)&u(k+p)&\cdots&u(k+p+N-2)
    \end{bmatrix}\\
    &\quad (13)
\end{align*}

The integer $N$ must be sufficiently large so that the rank of $Y_p(k)$ and $U_p(k)$ is at least equal to the rank of $\mathcal{O}_p$. 
Equation (12) is the key equation used to solve for $\mathcal{Q}_p$ and $\mathcal{T}_p$ and includes the input-and output-data information up to the data point $k+p+N-2$. 
Because the data matrix $Y_p(k)$ and $U_p(k)$ are the only information given, it is necessary to focus on these two matrices to extract information necessary to determine the system matrices $A,B,C,$ and $D$. 

The following quantities are defined as: 
\begin{align*}
    \mathcal{R}_{yy}&=\frac{1}{N}Y_p(k)Y_p^\top(k)\\
    \mathcal{R}_{yu}&=\frac{1}{N}Y_p(k)U_p^\top(k)\\
    \mathcal{R}_{uu}&=\frac{1}{N}U_p(k)U_p^\top(k)\\
    \mathcal{R}_{xx}&=\frac{1}{N}X(k)X^\top(k)\\
    \mathcal{R}_{yx}&=\frac{1}{N}Y_p(k)X^\top(k)\\
    \mathcal{R}_{xu}&=\frac{1}{N}X(k)U_p^\top(k)\\
    &\quad (14)
\end{align*}

where $N=\ell-p$, with $\ell$ being the data length and $p$ the data shift. 
The quantities $\mathcal{R}_{yy}$, $\mathcal{R}_{uu}$, and $\mathcal{R}_{xx}$ are symmetric matrices. 
The square matrices $\mathcal{R}_{yy}$ $(mp\times mp)$, $\mathcal{R}_{uu}$ $(rp\times rp)$, and $\mathcal{R}_{xx}$ $(n\times n)$ are the auto-correlations of the output data $y$ with time shifts, the input data $u$ with time shifts, and the state vector $x$, respectively. 
The rectangular matrices $\mathcal{R}_{yu}$ $(mp\times rp)$, $\mathcal{R}_{yx}$ $(mp\times n)$, and $\mathcal{R}_{xu}$ $(n\times rp)$ represent the cross correlations of the output data $y$ and the input data $u$, the output data $y$ and the state vector $x$, and the state vector $x$ and input data $u$, respectively. 
When the integer $N$ is sufficiently large, the quantities defined in equations (14) approximate expected values in the statistical sense if the input and output data are stationary processes satisfying the ergodic property. 

Taking the singular-value decomposition of the symmetric matrix $\mathcal{R}_{hh}$ yields 
\[
    \mathcal{R}_{hh}=\mathcal{U}\Sigma^2\mathcal{U}^\top
    =\begin{bmatrix}
        \mathcal{U}_n&\mathcal{U}_o
    \end{bmatrix}
    \begin{bmatrix}
        \Sigma_n^2&0_{n\times n_o}\\
        0_{n_o\times n}&o_{n_o}
    \end{bmatrix}
    \begin{bmatrix}
        \mathcal{U}_n^\top\\
        \mathcal{U}_o^\top
    \end{bmatrix}
    =\mathcal{U}_n\Sigma_n^2\mathcal{U}_n^\top
    \quad (26)
\]

The integer $n_o=pm-n$ is the number of dependent columns in $\mathcal{R}_{hh}$, $0_{n\times n_o}$ is an $n\times n_o$ zero matrix, and $0_{n_o}$ is a square-zero matrix of order $n_o$. 
The $pm\times n$ matrix $\mathcal{U}_n$ corresponds to the $n$ nonzero singular values in the diagonal matrix $\Sigma_n$, and the $pm\times n_o$ matrix $\mathcal{U}_o$ is associated with the $n_o$ zero singular values. 

\textbf{Partial decomposition method}. 
Regardless of which integer $p$ is chosen, the minimum value of $n_o$ must be $m$ (the number of outputs) to make $n<pm$, which will then satisfy the equality constraint in equation (27). 
There is one way of avoiding any singular-values truncation. 
Instead of taking the singular-value decomposition of the $pm\times pm$ square matrix $\mathcal{R}_{hh}$, factor only part of the matrix as follows: 
\[
    \mathcal{R}_{hh}(:,1:(p-1)m)=\mathcal{U}\Sigma^2\mathcal{V}^\top 
    =\begin{bmatrix}
        \mathcal{U}_n'&\mathcal{U}_o'
    \end{bmatrix}
    \begin{bmatrix}
        \Sigma_n^2&0_{n\times n_0}\\
        0_{n_o'\times n}&0_{n_o'\times n_o}
    \end{bmatrix}
    \begin{bmatrix}
        \mathcal{V}_n^\top\\
        \mathcal{V}_o^\top
    \end{bmatrix}
    =\mathcal{V}_n\Sigma_n^2\mathcal{V}_n^\top
    \quad (31)
\]

The dimension of $\mathcal{R}_{hh}(:,1:(p-1)m)$ is $pm\times (p-1)m$, meaning there are more rows than columns. 
The integer $n_o$ indicates the number of zero singular values and also the number of columns of $\mathcal{V}_o$. 
The integer $n_o'$ is the number of columns of $\mathcal{U}_o'$ that are orthogonal to the columns of $\mathcal{U}_n'$. 
For noisy data, there are no zero singular values, that is, $n_o=0$. 
If no singular values are truncated, $n_o'=m$ is obtained. 
If some small singular values are truncated, $n_o'$ becomes the sum of $m$ and the number of truncated singular values. 
Stated differently, there are at least $m$ columns of $\mathcal{U}_o'$ that are orthogonal to the columns of $\mathcal{U}_n'$ in equation (31). 

Equations (37) can be rewritten in the following matrix form: 
\[
    \mathcal{U}_{o\mathcal{T}}=\mathcal{U}_{on}\begin{bmatrix}
        D\\B
    \end{bmatrix}\quad (38)
\]
where 
\begin{align*}
    \mathcal{U}_{o\mathcal{T}}&=\begin{bmatrix}
        \mathcal{U}_o^\top\mathcal{T}_p(:,1:r)\\
        \mathcal{U}_o^\top\mathcal{T}_p(:,r+1:2r)\\
        \mathcal{U}_o^\top\mathcal{T}_p(:,2r+1:3r)\\
        \vdots\\
        \mathcal{U}_o^\top\mathcal{T}_p(:,(p-1)r+1:pr)
    \end{bmatrix}\\
    \mathcal{U}_{on}&=\begin{bmatrix}
        \mathcal{U}_o^\top(:,1:m)&\mathcal{U}_o^\top(:,m+1:pm)\mathcal{U}_n(1:(p-1)m,:)\\
        \mathcal{U}_o^\top(:,m+1:2m)&\mathcal{U}_o^\top(:,2m+1:pm)\mathcal{U}_n(1:(p-2)m,:)\\
        \mathcal{U}_o^\top(:,2m+1:3m)&\mathcal{U}_o^\top(:,3m+1:pm)\mathcal{U}_n(1:(p-3)m,:)\\
        \vdots&\vdots\\
        \mathcal{U}_o^\top(:,(p-1)m+1:pm)&0_{n_o\times n}
    \end{bmatrix}
\end{align*}

The dimension of $\mathcal{U}_{o\mathcal{T}}$ is $pn_o\times pr$ and the dimension of $\mathcal{U}_{on}$ is $pn_o\times (m+n)$. 
Let the right side of equation (34) be denoted by 
\[
    \mathcal{U}_{o\mathcal{R}}=\mathcal{U}_o^\top\mathcal{R}_{yu}\mathcal{R}_{uu}^{-1}\quad (39)
\]
where $\mathcal{U}_{o\mathcal{R}}$ is an $n_o\times pr$ matrix. 
Equation (38) shows that $\mathcal{U}_{o\mathcal{T}}$ is thus given by 
\[
    \mathcal{U}_{o\mathcal{T}}=\begin{bmatrix}
        \mathcal{U}_{o\mathcal{R}}(:,1:r)\\
        \mathcal{U}_{o\mathcal{R}}(:,r+1:2r)\\
        \mathcal{U}_{o\mathcal{R}}(:,2r+1:3r)\\
        \vdots\\
        \mathcal{U}_{o\mathcal{R}}(:,(p-1)r+1:pr)
    \end{bmatrix}
    \quad (40)
\]
and matrices $B$ and $D$ can be computed by 
\[
    \begin{bmatrix}
        D\\B
    \end{bmatrix}
    =\mathcal{U}_{on}^\dagger\mathcal{U}_{o\mathcal{T}}
    \quad (41)
\]
The first $m$ rows of $\mathcal{U}_{on}^\dagger\mathcal{U}_{o\mathcal{T}}$ form the matrix $D$, and the last $n$ rows produce the matrix $B$. 

Similar to equation (38), equation (52) can be rewritten in the following matrix form: 
\[
    \mathcal{O}_{p\Gamma}=\mathcal{O}_{pA}\begin{bmatrix}
        D\\B
    \end{bmatrix}
    \quad (53)
\]
where 
\footnotesize
\begin{align*}
    \mathcal{O}_{p\Gamma}&=
    \begin{bmatrix}
        \mathcal{O}_p^\dagger\Gamma(:,1:r)\\
        \mathcal{O}_p^\dagger\Gamma(:,r+1:2r)\\
        \mathcal{O}_p^\dagger\Gamma(:,2r+1:3r)\\
        \vdots\\
        \mathcal{O}_p^\dagger\Gamma(:,pr+1:(p+1)r)
    \end{bmatrix}\\
    \mathcal{O}_{pA}&=
    \begin{bmatrix}
        -A\mathcal{O}_p^\dagger(:,1:m)&
        I_n-A\mathcal{O}_p^\dagger(:,m+1:pm)\mathcal{O}_p(1:(p-1)m,:)\\
        \mathcal{O}_p^\dagger(:,1:m)-A\mathcal{O}_p^\dagger(m+1:2m,:)&
        \mathcal{O}_p^\dagger(:,m+1:pm)\mathcal{O}_p(:,1:(p-1)m)-A\mathcal{O}_p^\dagger(:,2m+1:pm)\mathcal{O}_p(1:(p-2)m,:)\\
        \mathcal{O}_p^\dagger(:,m+1:2m)-A\mathcal{O}_p^\dagger(2m+1:3m,:)&
        \mathcal{O}_p^\dagger(:,2m+1:pm)\mathcal{O}_p(:,1:(p-2)m)-A\mathcal{O}_p^\dagger(:,3m+1:pm)\mathcal{O}_p(1:(p-3)m,:)\\
        \vdots&\vdots\\
        \mathcal{O}_p^\dagger(:,(p-1)m+1:pm)&0_n
    \end{bmatrix}
\end{align*}
\normalsize
Here, $I_n$ is an identity matrix of order $n$ and $0_n$ is a zero matrix of order $n$. 
The quantity $\mathcal{O}_{p\Gamma}$ is a $pn\times r$ matrix and $\mathcal{O}_{pA}$ is a $(p+1)n\times (m+n)$ matrix. 

Substituting equations (62) into equation (57) yields 
\[
    y_N(0)=\Phi\Theta\quad (63)
\]
where 
\[
    \Theta=\begin{bmatrix}
        x(0)\\
        \underline{d}\\
        \underline{b}
    \end{bmatrix}\quad 
    \Phi=\begin{bmatrix}
        C&\underline{\mathcal{U}}_m(0)&0_{m\times n}\\
        CA&\underline{\mathcal{U}}_m(1)&C\underline{\mathcal{U}}_n(0)\\
        CA^2&\underline{\mathcal{U}}_m(2)&CA\underline{\mathcal{U}}(0)+C\underline{\mathcal{U}}_n(1)\\
        \vdots&&\\
        CA^{N-1}&\underline{\mathcal{U}}_m(N-1)&\sum_{k=0}^{N-2}CA^{N-k-2}\underline{\mathcal{U}}_n(k)
    \end{bmatrix}
    \quad (64)
\]
The vector size $\Theta$ is $(n+mr+nr)\times 1$ and the matrix size $\Phi$ is $mN\times (n+mr+nr)$. 
The unknown vector $\Theta$ can then be solved by 
\[
    \Theta=\Phi^\dagger y_N(0)\quad (65)
\]
where $\dagger$ denotes the pseudo-inverse. 